\documentclass[11pt, a4paper]{exam}
\usepackage[utf8]{inputenc}
\usepackage{amsmath}
\usepackage{siunitx}
\usepackage{fixltx2e}
\usepackage{tikz}

\begin{document}
	\begin{questions}
		\question
		The change of thick and thin disk star populations, with respect to radial distance from the Galaxy Center, on the Milky Way disk can be expressed using the following density model.
		\[
		\rho_{i}(z) = n_{i} \exp{\left( -\frac{|z|}{H_{i} \right)}}
		\]
		
		In this equation $i = 1$ and $i = 2$ express the thick and thin disks respectively, $z$ expresses the radial distances of the stars to the Galaxy plane, $\rho_{i} (z)$ expresses the density of the thin or thick disks with respect to radial distance, $n_i$ expresses the thin or thick disk population densities near the Sun, $H_i$ expresses the ratio for the exponential decrease in thin or thick disk population.
		
		Let us assume that the ratio of the exponential decrease for thin disks is 275 pc and for thick disks is 850 pc, and that the ratio of the thick disk population to the thin disk near the Sun is 0.08.
	
		\begin{parts}
			\part
			What is the radial distance where the thin and thick disk stars have the same density?
			
			\part
			What is the thin disk population density for the radial distance found in (a)?
			
			\[
				***
			\]
			
			\part
			Sun's distance to the center of the Milky Way is approximately 8.5 kpc and its tangential velocity about the Galaxy center is 220 km s$^{-1}$. Assuming that the majority of the mass is in a radius of 8.5 kpc, calculate the mass of the Milky Way.
			
			\part
			The Tully-Fisher Relation hypothesizes a relationship between the absolute s and maximum rotational velocities of spiral galaxies. The relative magnitude of a Cepheid variable in the galaxy NGC 3627 (Type: SBb) with a pulsation period of 41 days is being observed as $m = 22$. Knowing that the relative magnitude of the galaxy is $m\textsubscript{gal} = 8.92$ and that dispersion of light is negligible, calculate the maximum rotational velocity ($V$\textsubscript{max}) of NGC 3627.
			
			\[
				\text{Period Luminosity Relation (PLR):} \quad M = -2.43 \times \log_{10}P - 4.05
			\]
			\[
				\text{Tully-Fisher Relation (TFR):} \quad M\textsubscript{gal}=-10.2 \times \log_{10}V\textsubscript{max} + 2.71
			\]
			
			\part
			A supermassive black hole named Sgr A*, with a Schwarzschild radius of R\textsubscript{S} = $0.08$ au, is located at the center of the universe. In the galaxy center, a star named S2 is in the orbit nearest to Sgr A*. If the semi-major axis of the orbit of S2 is taken as $a$, what must be the ratio $a / R\textsubscript{S}$ for S2 to have an approximate orbital period of 16 years?
		\end{parts}
		
		\question
		Assume that you are observing a spacecraft in a circular orbit of radius 10$^5$ km around a distant planet and that you are observing in a plane collinear to the orbital plane of the spacecraft, from Earth. You have measured that the wavelength of the radio signals transmitted by the spacecraft periodically varies between 2.99964 m and 3.00036 m.
		
		\begin{parts}
			\part
			Assuming that the radio transmitter of this spacecraft is operating normally, find the constant wavelength of the stream.
			
			\part
			Find the orbital speed of the spacecraft relative to you.
			
			\part
			Find the orbital period of the spacecraft.
			
			\part
			Find the mass of this distant planet
			
			\part
			If the star of this planet has the strongest radiation in the wavelength of 579.6 nm, find the surface temperature of the star.
			
			\part
			If $R_\star / R_{sun} = 0.7$, find the total luminosity of the star in terms of $L_\odot$.
		\end{parts}
		
		\question
		The sun produces its energy through nuclear fusion reactions in its core. These reactions can be simply defined through the equations given below:
		
		\begin{center}
			$^1$H + $^1$H $\rightarrow$ $^2$H + $e^+$ + $\nu$
			\vskip
			$^2$H + $^1$H $\rightarrow$ $^3$He + $\gamma$
			\vskip
			$^3$He + $^3$He $\rightarrow$ $^4$He + $^1$H + $^1$H
		\end{center}
		
		The numbers above the symbols state the total amount of protons and neutrons in the nucleus, e$^+$ the positron, $\nu$ the neutrino and $\gamma$ the gamma radiation that is being emitted. As it can be seen from the equations, for the third step to occur, two $^3$He cores are required. For that, the first reaction must occur twice. The nuclear masses of the $^1$H and $^4$He cores are as; $1.6726 \times 10^{-27}$ kg and $6.6447 \times 10^{-27}$ kg, respectively.
		
		\begin{parts}
			\part
			Calculate the amount of mass that is converted to energy per reaction.
			
			\part
			If the luminosity of the Sun is known, calculate with explanations, the rate of the series of reactions required to produce this energy.
			
			\part
			How is the energy produced at the core of the Sun transferred to its surface?
			
			\part
			Explain the layers of the Solar Atmosphere, stating the temperature differences.
		\end{parts}
		
		\question
		The parallax and the proper motion of $\alpha$ Lyr (Vega) which is a Delta Scuti variable are 130 mas (milliarcseconds) and 240 mas respectively.
		
		\begin{parts}
			\part
			Calculate the tangential velocity of Vega.
			
			\part
			The average radial velocity of Vega is calculated as $-20.6$ km s$^{-1}$, calculate the space velocity of the star.
			
			\part
			If the laboratory wavelength of the Hydrogen-$\alpha$ (Balmer) line is 6563 \AA, at what wavelength in the spectra of Vega will the line be observed?
			
			\part
			Explain how the measured radial velocities will change due to the variation type of the star, draw a model for your explanation.
		\end{parts}
		
		
	\end{questions}
\end{document}